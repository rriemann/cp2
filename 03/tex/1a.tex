\section*{Aufgabe 3.1a)}
\label{sec:1a}
Im ersten Teil der Aufgabe war zu untersuchen, wie sich die logistische
Abbildung $$M(x) = rx\cdot (1-x)$$ für $r$-Werte von 4, 3.9 und 3.832 verhält.
Im Speziellen wurde das invariante Maß $ϱ(x)$ untersucht, indem für 1000 zu
iterierende Punkte die erreichten $x$-Werte histogrammiert wurden. Der
Quelltext, der dies umsetzt, ist in \lref{logabb} dargestellt.

\lstinputlisting[firstline=1,firstnumber=1,label=lst:logabb,caption={logabb.m}]{
../logabb.m}

Die hierin aufgerufenen Funktionen sind in \lref{logmap} und \lref{logmapp}
dargestellt.

\lstinputlisting[firstline=1,firstnumber=1,label=lst:logmap,caption={logmap.m}]{
../logmap.m}

\lstinputlisting[firstline=1,firstnumber=1,label=lst:logmapp,caption={logmapp.m}
]{../logmapp.m}

Die Vorlagen für die Quelltexte stammen aus der Vorlesung. 

Die resultierenden Plots sind in \fref{logabb} dargestellt.
\begin{figure}[htb]
\centering
  \includegraphics[width=1\columnwidth,keepaspectratio]{../logabb.png}
  \caption{Output des Programms}
  \label{fig:logabb}
\end{figure}

Im obersten Plot sind die erreichten $x$-Werte in Abhängigkeit von $r$
dargestellt. Wie man erkennen kann, werden für $r=3.9$ und $r=4$ im Rahmen der
Auflösung der Graphik alle $x$-Werte etwa 0.1 bzw. 0 und eins erreicht, für
$r=3.832$ dagegen werden nur 3 Punkte bzw. kleine, als Punkt erscheinende Werte
angenommen. Die möglichen $x$-Werte sind also nicht dicht im Phasenraum,
sondern nehmen nur drei diskrete Werte an. Dies äußert sich auch in den
anderen beiden Plots. Im mittleren ist der Exponent kleiner als null, was
zeigt, dass es sich hier um keine chaotische Bewegung handelt, und im
Histogramm (unterster Plot) kann ebenfalls festgestellt werden, dass ϱ nur in
drei Bins Werte ungleich null annimmt. Daher kann man für diesen Fall die
Zustandsdichte mit Hilfe von Dirac’schen δ-Distributionen direkt angeben:
\begin{eqnarray}
ϱ(x) &=& \lim_{N\to\infty}\frac{1}{N}\sum_{n=0}^{N}δ(x - M^{(n)}(x^0)
\end{eqnarray}

Es kann ausgenutzt werden, dass gilt:
\begin{eqnarray}
M^{(0)} &=& M^{(3)} = M^{(6)} = … \equiv x^0\\
M^{(1)} &=& M^{(4)} = M^{(7)} = … \equiv x^1\\
M^{(2)} &=& M^{(5)} = M^{(8)} = … \equiv x^2
\end{eqnarray}

Somit wird
\begin{eqnarray}
ϱ(x) &=& \lim_{N\to\infty}\frac{1}{N}\cdot\frac{N+1}{3}(δ(x - x^0) + δ(x
- x^1) + δ(x - x^2))\\
&=& \frac{1}{3}\cdot(δ(x - x^0) + δ(x - x^1) + δ(x - x^2)).
\end{eqnarray}

Für die anderen beiden $r$-Werte kann eine solche vereinfachte Form für ϱ nicht
angegeben werden, da hier prinzipiell fast jeder $x$-Wert erreicht werden kann.