\section*{Aufgabe 3.1a}
Im ersten Teil der Aufgabe war zu untersuchen, wie sich die logistische
Abbildung $$M(x) = rx\cdot (1-x)$$ für $r$-Werte von 4, 3.9 und 3.832 verhält.
Im Speziellen wurde das invariante Maß $ϱ(x)$ untersucht, indem für 1000 zu
iterierende Punkte die erreichten $x$-Werte histogrammiert wurden. Der
Quelltext, der dies umsetzt, ist in \lref{logabb} dargestellt.

\lstinputlisting[firstline=1,firstnumber=1,label=lst:logabb,caption={logabb.m}]{
../logabb.m}

Die hierin aufgerufenen Funktionen sind in \lref{logmap} und \lref{logmapp}
dargestellt.

\lstinputlisting[firstline=1,firstnumber=1,label=lst:logmap,caption={logmap.m}]{
../logmap.m}

\lstinputlisting[firstline=1,firstnumber=1,label=lst:logmapp,caption={logmapp.m}
]{../logmapp.m}

Die Vorlagen für die Quelltexte stammen aus der Vorlesung. 

Die resultierenden Plots sind in \fref{logabb} dargestellt.
\begin{figure}[htb]
\centering
  \includegraphics[width=1\columnwidth,keepaspectratio]{../logabb.png}
  \caption{Output des Programms}
  \label{fig:logabb}
\end{figure}