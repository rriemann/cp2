\section*{Aufgabe 3.2)}
In diesem Aufgabenteil war die Abbildung $$M(x) = q\cdot \sin(πx)$$ für $x\in
[0,1]$ und $0< q \le 1$ zu untersuchen. Dafür wurde ein neuer, dem aus
Aufgabe 3.1a) ähnelnder Quellcode geschrieben.

\lstinputlisting[firstline=1,firstnumber=1,label=lst:qsin,caption={qsin.m}]{
../qsin.m}

\lstinputlisting[firstline=1,firstnumber=1,label=lst:qsinfct,caption={qsinfct.m}
] {../qsinfct.m}

\lstinputlisting[firstline=1,firstnumber=1,label=lst:qsinabl,caption={qsinabl.m}
] {../qsinabl.m}

Zusätzlich zum Code aus der Aufgabe 3.1a) wird hier eine Schleife durchlaufen,
die einmal eine hohe und ein andermal eine wesentlich geringere
Rechengenauigkeit simuliert. 

Es resultieren also zwei Plots, die in \fref{qsin} dargestellt sind.

\begin{figure}[htb]%
\begin{center}%
  \subfloat[]{%
    \label{fig:genau}
    \includegraphics[width=0.9\textwidth]{../qsin0.png}}\\
  \subfloat[]{
    \label{fig:ungenau}
    \includegraphics[width=0.9\textwidth]{../qsin1.png}}
\caption{Untersuchung der Abbildung $q\sin(πx)$}
\label{fig:qsin}
\end{center}
\end{figure}

Wie man erkennen kann, stimmen die erreichten $x$-Werte für die beiden
Genauigkeiten annähernd überein. Auch die Verläufe der λ-Werte stimmen überein.
Einzig die Histogramme unterscheiden sich dadurch, dass durch die größere
Ungenauigkeit auch $x$-Werte von deutlich unter null erreicht werden können,
sodass negative Werte auftreten, die eigentlich nicht erwünscht sind. Offenbar
hat dies aber kaum Auswirkungen auf das chaotische Verhalten.

Dies ist erstaunlich, da chaotische Abbildung sehr sensitiv auf kleine
Veränderungen von Parametern sein sollte, was hier nicht bestätigt werden kann.
Demnach weist diese Abbildung in diesem Sinne kein klassisch chaotisches
Verhalten auf. Dennoch kommt es bei der Variation von $r$ zum chaotischen
Verhalten, sodass es sich bei der betrachteten Abbildung um eine chaotische
Abbildung handelt.