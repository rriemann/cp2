\section*{Aufgabe 3.2)}
In diesem Aufgabenteil war die Abbildung $$M(x) = q\cdot \sin(πx)$$ für $x\in
[0,1]$ und $0< q \le 1$ zu untersuchen. Dafür wurde ein neuer, dem aus
Aufgabe 3.1a) ähnelnder Quellcode geschrieben.

\lstinputlisting[firstline=1,firstnumber=1,label=lst:qsin,caption={qsin.m}]{
../qsin.m}

\lstinputlisting[firstline=1,firstnumber=1,label=lst:qsinfct,caption={qsinfct.m}
] {../qsinfct.m}

\lstinputlisting[firstline=1,firstnumber=1,label=lst:qsinabl,caption={qsinabl.m}
] {../qsinabl.m}

Zusätzlich zum Code aus der Aufgabe 3.1a) wird hier eine Schleife durchlaufen,
die einmal eine hohe und ein andermal eine wesentlich geringere
Rechengenauigkeit simuliert. 

Es resultieren also zwei Plots, die in \fref{qsin} dargestellt sind.

\begin{figure}[htb]%
\begin{center}%
  \subfloat[]{%
    \label{fig:genau}
    \includegraphics[width=0.8\textwidth]{../qsin0.png}}\\
  \subfloat[]{
    \label{fig:ungenau}
    \includegraphics[width=0.8\textwidth]{../qsin1.png}}
\caption{Untersuchung der Abbildung $q\sin(πx)$}
\label{fig:qsin}
\end{center}
\end{figure}
