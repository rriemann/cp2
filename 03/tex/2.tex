\section*{Aufgabe 3.2)}
In diesem Aufgabenteil war die Abbildung $$M(x) = q\cdot \sin(πx)$$ für $x\in
[0,1]$ und $0< q \le 1$ zu untersuchen. Dafür wurde ein neuer, dem aus
Aufgabe 3.1a) ähnelnder Quellcode geschrieben.

\lstinputlisting[firstline=1,firstnumber=1,label=lst:qsin,caption={qsin.m}]{
../qsin.m}

\lstinputlisting[firstline=1,firstnumber=1,label=lst:qsinfct,caption={qsinfct.m}
] {../qsinfct.m}

\lstinputlisting[firstline=1,firstnumber=1,label=lst:qsinabl,caption={qsinabl.m}
] {../qsinabl.m}

Zusätzlich zum Code aus der Aufgabe 3.1a) wird hier eine Schleife durchlaufen,
die einmal eine hohe und ein andermal eine wesentlich geringere
Rechengenauigkeit simuliert. 

Es resultieren also zwei Plots, die in \fref{qsin} dargestellt sind.

\begin{figure}[htb]%
\begin{center}%
  \subfloat[]{%
    \label{fig:genau}
    \includegraphics[width=0.9\textwidth]{../qsin0.png}}\\
  \subfloat[]{
    \label{fig:ungenau}
    \includegraphics[width=0.9\textwidth]{../qsin1.png}}
\caption{Untersuchung der Abbildung $q\sin(πx)$}
\label{fig:qsin}
\end{center}
\end{figure}

Wie man erkennen kann, stimmen die erreichten $x$-Werte für die beiden
Genauigkeiten annähernd überein. Auch die Verläufe der λ-Werte stimmen überein.
Einzig die Histogramme unterscheiden sich dadurch, dass durch die größere
Ungenauigkeit auch $x$-Werte von deutlich unter null erreicht werden können,
sodass negative Werte auftreten, die eigentlich nicht erwünscht sind. Auch
die konkreten $x$-Werte nehmen je nach Genauigkeit unterschiedliche Werte an.
Dazu muss angemerkt werden, dass es sich hierbei teilweise um einen Effekt der
Einteilung in Bins handeln könnte, da für die ungenauere Rechnung die zehn Bins
einen Bereich von $x \in [-1,1]$ abdecken, während die genauere Rechnung nur $x
\in [0,1]$ berücksichtigt.

Die Abbildung ist also sensitiv auf die Rechengenauigkeit, und auch bei der
Variation von $r$ kommt es zum chaotischen Verhalten, sodass es sich bei der
betrachteten Abbildung um eine chaotische Abbildung handelt.