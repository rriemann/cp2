\section*{Aufgabe 1d}
In den Kommentaren von \lref{eig_custom} sind Hinweise zur Schätzung des
Rechenaufwandes in Flops gegeben. Für ein groben Überblick werden nur die
Anzahl $N$ an Multiplikationen und Divisionen gezählt. Hierbei wird unter anderem die
Auswertung von trigonometrischen Funktionen vernachlässigt. Rechenschritte,
die nicht wenigstens mit der Matrixdimension $n$ skalieren, werden ebenfalls
vernachlässigt.

Die Anzahl der notwendigen Iterationsschritte $I$ ist a priori nicht feststellbar
und bleibt vorerst als Parameter in der Formel stehen.
\begin{eqnarray*}
 N &=& n^2 + I \sum^{n}_{q=2} \sum^{q-1}_{p=1} \left[ 2+2+2+1+1+n-2+2+2+1+1+2 \right] \\
 N &=& n^2 + I \sum^{n}_{q=2} (q-1)(14+n) \\
 N &=& n^2 + I \frac{n^2-n}{2}(14+n)
\end{eqnarray*}
Im Limes unendlich großer Matrizen erhalten wir
\begin{equation*}
 \lim_{n \to \infty} N = \frac{I}{2}n^3 = a n^3
\end{equation*}

Bei Variation von $n$ in den Grenzen von 10 bis hin zu 60, stellt man
fest, dass  $I = 6$ ein typischer Wert ist. Somit liegt der Faktor $a$
in der Größenordnung von 3.