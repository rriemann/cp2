\section*{Quelltext}
\lstinputlisting[firstline=1,firstnumber=1,label=lst:eig_custom,caption={eig\_custom.m}]{../eig_custom.m}

\section*{Beispielaufruf}
Zunächst wird in \lref{1} eine zufällige $5\times 5$-Matrix $A$ erzeugt.
\begin{lstlisting}[label=lst:1]
octave:2> A = rand(5)
A =

   0.297534   0.777789   0.664182   0.415014   0.269915
   0.983294   0.115236   0.971651   0.263736   0.868703
   0.564986   0.231347   0.975142   0.624327   0.553026
   0.850511   0.754599   0.929670   0.898320   0.598623
   0.083116   0.921160   0.806831   0.434421   0.425904
\end{lstlisting}
In \lref{2} wird die in \lref{eig_custom} definierte Funktion
\texttt{eig\_custom} aufgerufen. Innerhalb dieser Funktion wird $A$ zunächst
symmetrisch gemacht, so denn $A$ es noch nicht ist.

Vor der Rückgabe der berechneten Eigenwerte, werden noch einige zusätzliche
Ergebnisse zur Überprüfung ausgegeben.
\begin{itemize}
 \item Die Anzahl der in diesem Beispiel für Maschinengenauigkeit benötigten Iterationen:\\
       4
 \item Obwohl das Jacobi-Verfahren es nicht vorsieht, wird zusätzlich zur Berechnung
       der Eigenwerte noch die Matrix $V$ berechnet, die die Gesamtdrehung zur
       Diagonalisierung von $A$ beschreibt.
       \[ D = V^TAV \]
       Die Gültigkeit dieser Gleichung wird verifiziert. Man kann erkennen, dass
       die Nichtdiagonal-Elemente von \texttt{V'*A*V} in der Größenordnung von
       $10^{-16}$ liegen, welche das Limit der Maschinengenauigkeit darstellt.
 \item Weiterhin wird die Orthogonalität der Drehmatrix $V$ geprüft. Mit
       \[ V^TV = E \]
       konnte verifiziert werden, dass $V$ in der Tat orthogonal ist.
 \item Bevor dann der sortierte Spaltenvektor mit den gefundenen Eigenwerten
       $λ_i$ zurückgegeben wird, wird das Ergebnis zunächst noch mit den
       Eigenwerten aus der mitgelieferten Funktion \texttt{eig} verglichen. Die
       Werte unterscheiden sich lediglich im Rahmen der Maschinengenauigkeit.
\end{itemize}

\begin{lstlisting}[label=lst:2]
octave:3> eig_custom(A)
Iterationen:
 4
V'*A*V:
   3.1059e+00  -2.1461e-13   1.2490e-16  -3.3307e-16  -6.6613e-16
  -2.1452e-13  -9.6691e-01   2.1858e-16  -1.3878e-16  -1.1102e-16
   1.0582e-16   2.7148e-16   1.7491e-01   5.6379e-17  -3.0358e-17
  -2.3766e-16  -1.5959e-16   7.5460e-17   8.0324e-02  -7.2858e-17
  -4.0246e-16  -1.2490e-16  -1.9082e-17  -8.3267e-17   3.1787e-01
V'*V:
   1.00000   0.00000  -0.00000  -0.00000   0.00000
   0.00000   1.00000  -0.00000   0.00000   0.00000
  -0.00000  -0.00000   1.00000  -0.00000  -0.00000
  -0.00000   0.00000  -0.00000   1.00000  -0.00000
   0.00000   0.00000  -0.00000  -0.00000   1.00000
res-eig(A):
   0.0000e+00
  -8.3267e-17
   0.0000e+00
   2.2204e-16
  -4.4409e-16
ans =

  -0.966908
   0.080324
   0.174906
   0.317873
   3.105941
\end{lstlisting}
