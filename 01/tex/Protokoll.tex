\documentclass[a4paper,oneside,bibtotoc,smallheadings,pointlessnumbers,
halfparskip,DIV15]{scrartcl}
% tocleft,
\usepackage[pdftex]{graphicx}
\usepackage[ngerman]{babel}
\usepackage[utf8]{inputenc}
\usepackage{../assets/uniinput}
\usepackage[T1]{fontenc}
\usepackage{amssymb,amsmath}
\usepackage{booktabs}
\usepackage{enumerate}
\usepackage{subfig}
\usepackage{siunitx}
\sisetup{
	seperr		=	true,
	trapambigerr	=	true,
	openerr		=	(,
	closeerr	=	),
	expproduct	=	cdot,
	padnumber	=	both,
	stickyper	=	true,
	per		=	reciprocal,
	trapambigfrac	=	true,
	repeatunits	=	false,
	openfrac	=	(,
	closefrac	=	),
	prefixsymbolic 	=	true,
	prefixproduct	=	cdot,
	decimalsymbol	=	comma,
        tabnumalign     =       left,
        tabtextalign    =       left
}
\usepackage{color}
\definecolor{grey}{rgb}{.4,.4,.4}
\definecolor{darkgreen}{rgb}{0,.35,0}
\definecolor{ltgray}{gray}{0.90}
% \definecolor{darkblue}{rgb}{0,0,.6}
% \definecolor{darkred}{rgb}{.6,0,0}
% \definecolor{red}{rgb}{.98,0,0}
\usepackage{listings}
\lstdefinelanguage{Maxima}{
  keywords={addrow,addcol,zeromatrix,ident,augcoefmatrix,ratsubst,diff,ev,tex,%
    with_stdout,nouns,express,depends,load,submatrix,div,grad,curl,%
    rootscontract,solve,part,assume,sqrt,integrate,abs,inf,exp,float,log},
  sensitive=true,
  comment=[n][\itshape]{/*}{*/}
}
\lstset{language=C++,
%   commentstyle=\itshape\color{darkgreen},
%   commentstyle=\color{darkgreen},
%   keywordstyle=\bfseries, %\color{darkblue},
%   stringstyle=\color{darkred},
%   basicstyle=\ttfamily\scriptsize,
  morekeywords={TH1F,TLorentzVector,TVector3,vector,TFile,TFitResultPtr,TF1,\
                TGraph,TH1,TObject,TCanvas,string,Double_t,TGraphErrors},
  basicstyle=\scriptsize,
  numbers=left,
  numberstyle=\tiny,%\color{gray},
  stepnumber=1,
  tabsize=4,
  showspaces=false,
  showstringspaces=false,
  breaklines=true,
  frame=lrtb,
  captionpos=b,
  extendedchars=true,
  inputencoding=utf8,
%   backgroundcolor=\color{ltgray}
}
\usepackage[pdftex]{hyperref}
\hypersetup{
% 	colorlinks	=	true,
% 	urlcolor	=	darkblue,
	pdftitle	=	{Protokoll zum F-Praktikumsversuch 'Compton-Effekt'},
	pdfsubject	=	{Praktikumsversuch}
	pdfauthor	=   	{robert.riemann@physik.hu-berlin.de,thomas.murach@physik.hu-berlin.de},
	pdfkeywords	=	{Fortgeschrittenen-Praktikum,Praktikumsversuch, 2010, Compton-Effekt}
% 	pdfcreator	=	{pdftex},
% 	pdfproducer	=	{pdftex}
}

\newcommand{\dd}[1]{\mathrm{d}#1\,} % declare dx operator, usage: \dd{x} for dx
\newcommand{\lref}[1]{Listing (\ref{lst:#1})} % refer to a listing, usage: \lref{label} for Listing (...)
\newcommand{\fref}[1]{Abb. (\ref{fig:#1})} % refer to a figure, usage: \lref{label} for Abb. (...)
\newcommand{\tref}[1]{Tab. (\ref{tab:#1})} % refer to a table, usage: \lref{label} for Tab. (...)
\newcommand{\eref}[1]{Gl. (\ref{eqn:#1})} % refer to a equation, usage: \lref{label} for Tab. (...)


\begin{document}
% % % % % % % % % % % % % % % % % % % % % % % % 
\title{{\centering \rule{15cm}{0.001cm}\\
\Large{\textsc{Institut für Physik der
Humboldt-Universität zu Berlin}}}\\ \centering \rule{15cm}{0.001cm}\\
\vspace{15mm} \centering
\includegraphics[scale=0.9]{../assets/siegel}\\
\vspace{18mm}
{\bf{\huge{Computational Physics II}}}\\
Übungsblatt 1\\
% \vspace{14mm}
% Compton-Effekt\\
% \vspace{14mm} {\small{\textbf{Betreuer: M. zur Nedden}}}\\}
}
\author{Robert Riemann; Matr.Nr.: 521085\\
Thomas Murach; Matr.Nr.: 517771\vspace{18mm}}
\vspace{18mm}
% \date{15. Juni 2008}
% % % % % % % % % % % % % % % % % % % % % % % %
% \onecolumn
\maketitle
% \twocolumn

\tableofcontents
\listoffigures
\listoftables

Nun sollen die in der Aufgabenstellung gegebenen Relationen
\begin{eqnarray}

\end{eqnarray}
bewiesen werden.

% \input{auswertung}
% \input{ergebnis}
% \input{quellen}
% \input{anhang}

% \section{Beispiele}
% blabla
% 
% Weitere Information sind der Versuchsbeschreibung im \cite{script} zu entnehmen.
% 
% \begin{eqnarray}
% 	g &=& \SI{9.8157(11)}{\metre\per\second\squared} \\
%         \left[ g \right] &=& \si{\metre\per\Square\second} % großschreibung!
% 	\label{eqn:asdf}
% \end{eqnarray}
% 
% Dann kann man auf \eref{asdf} verweisen.


% \begin{figure}[htb]
% 	\centering
% 	\includegraphics[width=1\columnwidth,keepaspectratio]{Winkelabhaengigkeit2}
% 	\caption{Periodendauer in Abhängigkeit der Maximalauslenkung}
% 	\label{fig:Winkelabhaengigkeit}
% \end{figure}

% \begin{table}[htbp]
% \centering
% \setlength{\tabcolsep}{14pt}
% \begin{tabular*}{\columnwidth}{%
% S[tabformat=2.1]%
% S[tabformat=2.2]%
% S[tabformat=1.2]}
% \toprule
% {$R$ in \si{\ohm}} &
% {$U$ in \si{\volt}} &
% {$\frac{U}{U_{R=0}}$}\\
% \midrule
% \multicolumn{3}{c}{\textit{Eingangswiderstand $R_E$}}\\
% \midrule
% 0 & 16 & 1 \\
% 4.7e3 & 8 & 0.5 \\
% \midrule
% \multicolumn{3}{c}{\textit{Ausgangswiderstand $R_A$}}\\
% \midrule
% 0 & 1.8 & 1 \\
% 12 & 0.52 & 0.29 \\
% 27 & 1.2 & 0.67 \\
% 18 & 0.8 & 0.44 \\
% \bottomrule
% \end{tabular*}
% \label{tab:221messungexp_widerstaende}
% \caption{experimentelle Messung der Ein- und Ausgangswiderstände}
% \end{table}


% \newpage

% \onecolumn
% \appendix
% \begin{figure}
% 	\centering
% 	\includegraphics[width=0.98\textwidth,keepaspectratio]{Messprotokoll}
% 	\caption{Messprotokoll}
% 	\label{fig:protokoll}
% \end{figure}
%\colorbox{yellow}{} Farben verwenden
\end{document}