\section*{Aufgabe 1c}

Nun sollen die in der Aufgabenstellung gegebenen Relationen
\begin{eqnarray}
v'_{ip} &=& v_{ip} - s(v_{iq} + τv_{ip})\\
v'_{iq} &=& v_{iq} + s(v_{ip} - τv_{iq})
\end{eqnarray}
bewiesen werden, wobei $v_{ab}$ bzw. $v'_{ab}$ die $(a,b)$-Komponente der
Matrizen $V$ bzw. $V'$ sind. Um diese Relationen zu erhalten, muss der
konkrete Aufbau der Matrizen $P$ bekannt sein. Diese bestehen aus einer
Einheitsmatrix, bei der einzig die Elemente in der $p$- bzw. $q$-ten Zeile bzw.
Spalte verändert werden. Dies lässt sich wie folgt darstellen:
\begin{equation}
p_{ij} = \begin{cases}\begin{array}[h]{ll}
1 & \mathrm{f"ur~} i=j \notin \{p,q\} \\c & \mathrm{f"ur~} i=j \in \{p,q\}\\s &
\mathrm{f"ur~} i=p, j=q\\-s & \mathrm{f"ur~} i=q, j=p\\0 & \mathrm{sonst}
\end{array}\end{cases}
\end{equation}
% auch interessant: (aber nur außerhalb der mathe-umgebung
% \begin{tabbing}
% 1 \= \quad \= iae\\c \> \> iaee\\s \> \> aie\\-s \> \> ie\\0 \> \> ieie
% \end{tabbing}

In dieser Gleichung für $p_{ij}$ stehen $c$ bzw. $s$ für den Kosinus bzw. Sinus
des Drehwinkels φ:
\begin{eqnarray}
c &=& \cos(φ)\\
s &=& \sin(φ)
\end{eqnarray}

Um die Definition der $P$-Matrizen in eine kompaktere Form zu bringen, wird nun
eine Indexschreibweise mit Hilfe der Kronecker-Deltas $δ_{ij}$ verwendet:
\begin{eqnarray}
p_{ij} &=& δ_{ij} + s\cdot δ_{ip}δ_{jq} - s\cdot δ_{iq}δ_{jp} + (c-1)\cdot
δ_{ip}δ_{jp} + (c-1)\cdot δ_{iq}δ_{jq}
\end{eqnarray}
Von den beiden Diagonalelementen muss jeweils eine eins abgezogen werden, um
die durch den $δ_{ij}$-Term zu kompensieren. Wenn man nun eine „beliebige“
Matrix $V$ mit $P$ multipliziert, ergeben sich (unter Verwendung der
Einsteinschen Summenkonvention) die gesuchten Relationen:
\begin{eqnarray}
v'_{kj} &=& v_{ki} P_{ij} = v_{ki}δ_{ij} + s\cdot v_{ki}δ_{ip}δ_{jq} - s\cdot
v_{ki}δ_{iq}δ_{jp} + (c-1)v_{ki}(δ_{ip}δ_{jp} + δ_{iq}δ_{jq})\\
&=& v_{kj} + s\cdot v_{kp}δ_{jq} - s\cdot v_{kq}δ_{jp} + (c-1)(v_{kp}δ_{jp} +
v_{kq}δ_{jq})
\end{eqnarray}

Nun kann eine Fallunterscheidung durchgeführt werden, die berücksichtigt,
welche der Kronecker-Deltas null bzw. eins werden.

\begin{eqnarray}
\begin{array}[h]{cclcccl}
j &\notin& {p,q} &:& v'_{kj} &=& v_{kj}\\
j &=& p &:& v'_{kp} &=& v_{kp} - s\cdot v_{kq} + (c-1)v_{kp}\\
&&&&&=& v_{kp} - s\cdot(v_{kq} - \frac{c-1}{s}v_{kp})\\
j &=& q &:& v'_{kq} &=& v_{kq} + s\cdot v_{kp} + (c-1)v_{kq}\\
&&&&&=& v_{kp} + s\cdot(v_{kq} + \frac{c-1}{s}v_{kp})
\end{array}
\label{eqn:vmatrix}
\end{eqnarray}

Schließlich wird gezeigt, dass man den jeweils auftretenden Bruch in die
gewünschte Form bringen kann:
\begin{eqnarray}
\frac{c-1}{s} &=& \frac{s\cdot(c-1)}{s^2} = \frac{s\cdot(c-1)}{1-c^2} =
\frac{s\cdot(c-1)}{(1-c)\cdot(1+c)}\\
&=& -\frac{s}{c+1} \equiv -τ 
\end{eqnarray}

Dieses Ergebnis wird nun in \eref{vmatrix} eingesetzt:
\begin{eqnarray}
v'_{kp} &=& v_{kp} - s\cdot(v_{kq} + τv_{kp})\\
v'_{kq} &=& v_{kp} + s\cdot(v_{kq} - τv_{kp})
\end{eqnarray}
Dies entspricht der gesuchten Lösung.