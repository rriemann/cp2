Nun sollen die in der Aufgabenstellung gegebenen Relationen
\begin{eqnarray}
v'_{ip} &=& v_{ip} - s(v_{iq} + τv_{ip}\\
v'_{iq} &=& v_{iq} + s(v_{ip} - τv_{iq}
\end{eqnarray}
bewiesen werden. Dafür muss der konkrete Aufbau der Matrizen $P$ bekannt sein.
Diese bestehen aus einer Einheitsmatrix, bei der einzig die Elemente in der
$p$- bzw. $q$-ten Zeile bzw. Spalte verändert werden. Dies lässt sich wie folgt
darstellen:
\begin{equation}
p_{ij} = \begin{cases}\begin{array}[h]{ll}
1 & \mathrm{f"ur~} i=j \notin \{p,q\} \\c & \mathrm{f"ur~} i=j \in \{p,q\}\\s &
\mathrm{f"ur~} i=p, j=q\\-s & \mathrm{f"ur~} i=q, j=p\\0 & \mathrm{sonst}
\end{array}\end{cases}
\end{equation}
% auch interessant: (aber nur außerhalb der mathe-umgebung
% \begin{tabbing}
% 1 \= \quad \= iae\\c \> \> iaee\\s \> \> aie\\-s \> \> ie\\0 \> \> ieie
% \end{tabbing}
