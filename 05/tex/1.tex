\section*{Aufgabe 5.1}
Im Folgenden wurde mittels der Fourier-Analyse eine Möglichkeit der Bildkomprimierung
getestet. Das in Matlab als $(m\times n)$-Matrix importierte Bild ohne Veränderungen ist
in \fref{orig} abgebildet. Die Matrix enthält die Graustufenwerte als Fließkommazahlen
(\texttt{double}).

\begin{figure}[htb]
\centering
  \includegraphics[width=0.7\textwidth,keepaspectratio]{../tmp/original}
  \caption{Orginales Bild ohne Komprimierung}
  \label{fig:orig}
\end{figure}

\section*{Aufgabe 5.1a)}

Zunächst wurde eine Funktion \lref{cut_rect} geschrieben, die den Rand des Graustufen-Bildes
schwärzt, die Pixel also auf 0 setzt. Die Breite wird hierbei in Abhängigkeit
des Kompressionsparameters $ρ \in ]0,1]$ so eingestellt, dass $hb\cdot ρ^2$
nicht-triviale Pixel verbleiben, wobei $h$ die Höhe und $b$ die Breite des
Bildes in Pixeln ist.

Wird die Funktion aus \lref{cut_rect} unter Verwendung von $ρ^2 = 0.4$ auf das
Bild in \fref{orig} angewendet, so erhält man die in \fref{1a} gezeigte Abbildung.
\lstinputlisting[firstline=3,lastline=13,label=lst:fft-test,caption={blatt5.m}]{../code/blatt5.m}
\lstinputlisting[firstline=1,lastline=10,firstnumber=1,label=lst:cut_rect,caption={cut\_rect.m}]{../code/cut_rect.m}

\begin{figure}[htb]
\centering
  \includegraphics[width=0.7\textwidth,keepaspectratio]{../tmp/eins_a}
  \caption{Bild mit Rand bei $ρ^2 = 0.4$}
  \label{fig:1a}
\end{figure}

Wie man erkennen kann, wurde abgesehen von der gewünschten Rahmen-Erzeugung
auch der innere Bereich des Bildes verändert, was allerdings nicht ein Effekt der
Beschneidung ist, sondern vermutlich auf die begrenzte Auflösung der Farbskala 
zurückzuführen ist. Daher sind beispielsweise die in der Bildmitte befindlichen
hellgrauen Blöcke im transformierten Bild nicht mehr zu finden. 

\section*{Aufgabe 5.1b)}
Für die folgenden Aufgaben wurden die in Matlab zur Verfügung stehenden 
Fast-Fourier-Trans\-formationsfunktionen \texttt{fft, ifft, fft2, ifft2} benötigt.
Zunächst wurde getestet, wie diese funktionieren, indem mit einfachen Vektoren und
Matrizen eine Hin- und eine Rücktransformation durchgeführt wurde, wie in 
\lref{fft-test} gezeigt.
\lstinputlisting[firstline=15,lastline=28,label=lst:fft-test,caption={blatt5.m}]{../code/blatt5.m}

Der zugehörige Output ist in \lref{fft-test-output} gezeigt.
\begin{lstlisting}[caption=Output des Beispielaufrufs,label=lst:fft-test-output]
a =

    0.0759
    0.0540
    0.5308
    0.7792
    0.9340


a_fft =

   2.3738          
  -0.6786 + 0.9830i
  -0.3186 + 0.2811i
  -0.3186 - 0.2811i
  -0.6786 - 0.9830i


a_2 =

    0.0759
    0.0540
    0.5308
    0.7792
    0.9340


a_fftshift =

    0.7792
    0.9340
    0.0759
    0.0540
    0.5308


a_ifftshift =

    0.0759
    0.0540
    0.5308
    0.7792
    0.9340


A =

    0.1299    0.1622    0.6020    0.4505    0.8258
    0.5688    0.7943    0.2630    0.0838    0.5383
    0.4694    0.3112    0.6541    0.2290    0.9961
    0.0119    0.5285    0.6892    0.9133    0.0782
    0.3371    0.1656    0.7482    0.1524    0.4427


A_fft =

  11.1456            -0.8578 + 0.2117i  -0.9221 + 1.6125i  -0.9221 - 1.6125i  -0.8578 - 0.2117i
  -0.5132 - 0.6404i   0.6573 - 1.2749i  -0.1080 + 1.1941i  -0.2687 + 0.1951i   0.3350 - 1.9203i
   0.3664 + 0.1807i  -1.9259 + 1.5904i  -1.8499 + 0.0229i   1.4279 - 0.0361i  -0.2900 - 0.2633i
   0.3664 - 0.1807i  -0.2900 + 0.2633i   1.4279 + 0.0361i  -1.8499 - 0.0229i  -1.9259 - 1.5904i
  -0.5132 + 0.6404i   0.3350 + 1.9203i  -0.2687 - 0.1951i  -0.1080 - 1.1941i   0.6573 + 1.2749i


A_2 =

    0.1299    0.1622    0.6020    0.4505    0.8258
    0.5688    0.7943    0.2630    0.0838    0.5383
    0.4694    0.3112    0.6541    0.2290    0.9961
    0.0119    0.5285    0.6892    0.9133    0.0782
    0.3371    0.1656    0.7482    0.1524    0.4427


A_fftshift =

    0.9133    0.0782    0.0119    0.5285    0.6892
    0.1524    0.4427    0.3371    0.1656    0.7482
    0.4505    0.8258    0.1299    0.1622    0.6020
    0.0838    0.5383    0.5688    0.7943    0.2630
    0.2290    0.9961    0.4694    0.3112    0.6541


A_ifftshift =

    0.1299    0.1622    0.6020    0.4505    0.8258
    0.5688    0.7943    0.2630    0.0838    0.5383
    0.4694    0.3112    0.6541    0.2290    0.9961
    0.0119    0.5285    0.6892    0.9133    0.0782
    0.3371    0.1656    0.7482    0.1524    0.4427
\end{lstlisting}

Wie man erkennen kann, sind die Vektoren und Matrizen nach zwei Transformationen
gleich den hineingesteckten Objekten, sodass die grundlegende Funktionalität als
gegeben angenommen werden kann.

Weiterhin wurden die Funktionen \texttt{fftshift} und \texttt{ifftshift}
untersucht (siehe \lref{fft-test}). Diese führen zunächst eine Hin- bzw. Rücktransformation
durch und verschieben die $k$-Werte in der Art, dass pro Zeile und Spalte die 
zu $k=0$ gehörenden Werte in die Mitte gelangen. Dies entspricht einer Symmetrisierung.
Die Funktion \texttt{ifftshift} macht diese Symmetrisierung wieder rückgängig, um
die ursprüngliche Matrix wiederzubekommen.

Die beiden Funktionen sollten sich nicht unterscheiden, da nur Blöcke der Matrizen
vertauscht und bei der Rücktransformation wieder zurückvertauscht werden sollten.
Wenn aber die Länge der Matrizen in einer oder zwei der Dimensionen ungerade ist,
dann ist diese einfache Blockvertauschung nicht mehr ohne weiteres durchführbar.
Daher wird eine etwas kompliziertere Operation durchgeführt, deren Inverses nicht
mehr sie selbst ist.


\section*{Aufgabe 5.1c)}
Zur Komprimierung des Bildes aus \fref{orig} wird eine 2D-Fourier-Transformation
in den $k$-Raum durchgeführt. Im Anschluss wird das Bild symmetrisiert, die Bildinformation
auf $hb\cdot ρ^2$ durch das Abschneiden des Randes reduziert, die Symmetrisierung wieder
rückgängig gemacht und zuletzt wird das Bild wieder durch die inverse Fourier-Trafo in den
$x$-Raum überführt.

Das Ergebnis aus \lref{1c} für $ρ=0.5$ ist in \fref{1c5} und für $ρ=0.1$ in \fref{1c1} dargestellt.
\lstinputlisting[firstline=30,lastline=37,label=lst:1c,caption={blatt5.m}]{../code/blatt5.m}

\begin{figure}[htb]
\centering
  \includegraphics[width=0.7\textwidth,keepaspectratio]{../tmp/eins_c_0_5}
  \caption{komprimiertes Bild mit rechteckigem Rand im $k$-Raum, $ρ=0.5$}
  \label{fig:1c5}
  \includegraphics[width=0.7\textwidth,keepaspectratio]{../tmp/eins_c_0_1}
  \caption{komprimiertes Bild mit rechteckigem Rand im $k$-Raum, $ρ=0.1$}
  \label{fig:1c1}
\end{figure}

Die Interpretation wird im nächsten Abschnitt durchgeführt.

\section*{Aufgabe 5.1d)}

In Aufgabe d) wurde equivalent zur Verfahrensweise in c) vorgegangen.
Jedoch wurde im transformierten Raum nicht ein rechteckiger Rand abgeschnitten,
sondern ein Kreisförmiger, siehe \lref{cut_round}. Hierbei wurde darauf geachtet, dass die verbleibende
Bildinformation nach wie vor $hb\cdot ρ^2$ ist. Der aufrufende Quelltext ist in \lref{k_kreis} und 
die aufgerufene Funktion \texttt{cut\_round} dargestellt.

\lstinputlisting[firstline=1,lastline=15,label=lst:cut_round,caption={cut\_round.m}]{../code/cut_round.m}
\lstinputlisting[firstline=39,label=lst:k_kreis,caption={blatt5.m}]{../code/blatt5.m}

Im Ergebnis erhalten wir \fref{1d5} mit $ρ=0.5$ und \fref{1d1} mit $ρ=0.1$.

\begin{figure}[htb]
\centering
  \includegraphics[width=0.7\textwidth,keepaspectratio]{../tmp/eins_c_0_5}
  \caption{komprimiertes Bild mit kreisförmigen Rand im $k$-Raum, $ρ=0.5$}
  \label{fig:1d5}
\end{figure}

\begin{figure}[htb]
\centering
  \includegraphics[width=0.7\textwidth,keepaspectratio]{../tmp/eins_c_0_1}
  \caption{komprimiertes Bild mit kreisförmigen Rand im $k$-Raum, $ρ=0.1$}
  \label{fig:1d1}
\end{figure}

Durch die Komprimierung, verändert sich das Bild. Das Ausmaß der Veränderung
hängt von der lokalen Struktur des Bildes ab. So kann man erkennen, dass
große einfarbige Flächen Substrukturen aufweisen, also nicht sehr gut erhalten
bleiben. Dagegen bleiben periodische Muster, wie z.\,B. das Gitter, sehr gut
erhalten. Jedoch verschwimmen auch hier etwas Grenzen.

Farbverläufen erkennt man nach der Komprimierung deutlich die diskreten Stufen an.

Die Ergebnisse können folgendermaßen erklärt werden: Flächen entsprechen einem Plateau,
welches für eine gute Annäherung auch sehr hohe Frequenzen benötigt, welche wir jedoch
abgeschnitten haben. Daher erkennt man periodische Substrukturen.

Das Gitter selbst ist periodisch und weißt nur sehr kurze Plateaus (Peaks) auf, die auf Grund
der periodischen Natur der Fourier-Transformation im $k$-Raum auch ohne hohe Frequenzen
gut wiedergegeben werden können.

Wird eine höhere Kompression gewählt, so werden die Substrukturen deutlicher.
Eine geringere Beschneidung im $k$-Raum führt also zu einer tendenziell größeren
Erhaltung der Informationen, was auch zu erwarten war.

Bei der eingestellten Auflösung sind keine Unterschiede zwischen den Bildern aus c) und d) zu
erkennen. Offenbar tragen für diese Auflösung also nur die Beträge wesentlich zur Kompression bei.

\section*{Aufgabe 5.1e)}

Momentan sind die Werte im $k$-Raum komplex. Es müssen also 2 Double-Werte pro Matrixelement
abgespeichert werden. Nach der Rückkonvertierung sind alle Matrix-Elemente wieder reell.

Es ist vorstellbar, dass die Matrix im $k$-Raum in eine Form gebracht wird, so dass deren Werte
rein reell sind. Es müsste dann nur noch ein Double-Wert pro Matrixelement abgespeichert werden.
Bei der Rückkonvertierung würde man per Konvention nur den Realteil der komplexen
Zahlen anzeigen.