\section*{Aufgabe 1}
\subsection*{1a)}
Zunächst soll der Ausdruck für die eindimensionale Zustandssumme $Z$ für das
Ising-Modell mit nicht verschwindendem Magnetfeld hergeleitet werden. Diese ist
nach dem Skript, Gl. (8.14), zu bestimmen als
\begin{eqnarray}
Z &=& \mathrm{Tr}\left[T^L\right]\\
\mathrm{mit\quad} T_{σσ'} &=& \exp[βσσ' + βB(σ+σ')/2]
\end{eqnarray}
σ und $σ'$ können hierbei $\pm 1$ sein. Damit ergibt sich die folgende Matrix $T$:
\begin{eqnarray}
T = \begin{pmatrix} e^{β(1+B)} & e^{-β} \\ e^{-β} & e^{β(1-B)} \end{pmatrix}
\end{eqnarray}

Nun sind die Eigenwerte dieser Matrix zu berechnen. Dies wird in den folgenden
Zeilen getan.

\begin{eqnarray}
0 &=& \det[T-λ] = (e^{β(1+B)}-λ)(e^{β(1-B)}-λ) - e^{-2β}\\
&=& λ^2 - λe^β\underbrace{\left(e^{βB} + e^{-βB}\right)}_{2\cosh(βB)} + \underbrace{e^{2β} - e^{-2β}}_{2\sinh(2β)}\\
→ λ_{\pm} &=& e^β\cosh(βB) \pm \sqrt{e^{2β}\cosh^2(βB) - 2\sinh(2β)}
\end{eqnarray}

Damit lässt sich die Potenzierung und anschließende Spurbildung der Matrix leicht
analytisch durchführen:
\begin{eqnarray}
\mathrm{Tr}[T^L] = λ_+^L + λ_-^L
\end{eqnarray}

$Z$ ist somit exakt bekannt. Um nun den Grenzwert $L→∞$ durchzuführen, wird ausgenutzt,
dass $λ_-$ kleiner als $λ_+$ ist. Daher ist der Beitrag von $λ_-$ im Grenzfall
vernachlässigbar.

\begin{eqnarray}
Z \approx λ_+^L = \left(e^β\cosh(βB) + \sqrt{e^{2β}\cosh^2(βB) - 2\sinh(2β)}\right)^L
\end{eqnarray}

\subsection*{1b)}
Hier soll aus der bekannten, kanonischen Zustandssumme die mittlere Energie und
Magnetisierung berechnet werden. Dafür wird die Relation aus Skript, Gl. (8.9) (mittlere
Energie) und Gl. (8.10) (mittlere Magnetisierung) verwendet. Die Relation für die mittlere
Energie $E$ lautet wie folgt:
\begin{eqnarray}
E &=& \langle H \rangle = -\frac{∂\ln Z}{∂β}\\
&=& {{\left({{2e^{2β}B\cosh \left(βB\right)\sinh \left(βB\right)+2e^{2β}\cosh^2
 \left(βB\right)-4\cosh \left(2β\right)}\over{2\sqrt{
 e^{2β}\cosh ^2\left(βB\right)-2\sinh \left(2β\right)}}}+e^β(B\sinh \left(βB\right)+
 \cosh \left(βB\right))\right)L}
 \over{\sqrt{e^{2β} \cosh^2\left(βB\right)-2\sinh \left(2β\right)}+e^{
 β}\cosh \left(βB\right)}}
\end{eqnarray}

Für die Magnetisierung $M$ ergibt sich folgendes:
\begin{eqnarray}
\langle M \rangle &=& \frac{1}{β}\frac{∂\ln Z}{∂B}\\
&=& {L\left({{e^{2β}\cosh \left(βB\right)\sinh \left(βB\right)}\over{\sqrt{e^{2β}\cosh^2
 \left(βB\right)-2\sinh \left(2β\right)}}}+e^β \sinh \left(βB\right)\right)}
 \over{\sqrt{e^{2β}\cosh ^2\left(βB\right)-2\sinh \left(2β\right)}+e^{β}\cosh \left(βB\right)}
\end{eqnarray}

\subsection*{1c}

Schließlich war die Magnetisierung in Abhängigkeit von β und $B$ zu plotten.
Dafür wurde das in \lref{plot_mag} dargestellte Skript geschrieben.

\lstinputlisting[label=lst:plot_mag,caption={magnetisierung.m}]{../code/magnetisierung.m}

Die daraus erstellte Graphik ist in \fref{magnetisierung} dargestellt.

\begin{figure}[htb]
  \centering
  \includegraphics[width=0.8\columnwidth,keepaspectratio]{../tmp/magnetisierung-crop}
  \caption{Magnetisierung in Abhängigkeit von β und $B$, jeweils in natürlichen Einheiten}
  \label{fig:magnetisierung}
\end{figure}

Das Ergebnis entspricht den Erwartungen, da zum Einen bei großen Magnetfeldern die Magnetisierung
ihren positiven oder negativen Grenzwert anstrebt, je nach Ausrichtung des Magnetfelds.
Weiterhin ist zu beobachten, dass bei kleinen Temperaturen, also großen β-Werten,
die Magnetisierung fast sofort ihren Grenzwert erreicht, während bei großen Temperaturen,
also kleinen β-Werten, auch große äußere Magnetfelder die Magnetisierung nicht 
wesentlich vom Wert $M=0$ abweichen lassen. Dies entspricht der Vorstellung, dass
eine hohe Temperatur die einzelnen Spins stärker aus der Ausrichtung in eine
Richtung ablenkt als eine geringe Temperatur.