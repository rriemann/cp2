\section*{Aufgabe 1}
\subsection*{1a)}
Zunächst soll der Ausdruck für die eindimensionale Zustandssumme $Z$ für das
Ising-Modell mit nicht verschwindendem Magnetfeld hergeleitet werden. Diese ist
nach dem Skript, Gl. (8.14), zu bestimmen als
\begin{eqnarray}
Z &=& \mathrm{Tr}\left[T^L\right]\\
\mathrm{mit\quad} T_{σσ'} &=& \exp[βσσ' + βB(σ+σ')/2]
\end{eqnarray}
σ und $σ'$ können hierbei $\pm 1$ sein. Damit ergibt sich die folgende Matrix $T$:
\begin{eqnarray}
T = \begin{pmatrix} e^{β(1+B)} & e^{-β} \\ e^{-β} & e^{β(1-B)} \end{pmatrix}
\end{eqnarray}

Nun sind die Eigenwerte dieser Matrix zu berechnen. Dies wird in den folgenden
Zeilen getan.

\begin{eqnarray}
0 &=& \det[T-λ] = (e^{β(1+B)}-λ)(e^{β(1-B)}-λ) - e^{-2β}\\
&=& λ^2 - λe^β\underbrace{\left(e^{βB} + e^{-βB}\right)}_{2\cosh(βB)} + \underbrace{e^{2β} - e^{-2β}}_{2\sinh(2β)}\\
→ λ_{\pm} &=& e^β\cosh(βB) \pm \sqrt{e^{2β}\cosh^2(βB) - 2\sinh(2β)}
\end{eqnarray}

Damit lässt sich die Potenzierung und anschließende Spurbildung der Matrix leicht
analytisch durchführen:
\begin{eqnarray}
\mathrm{Tr}[T^L] = λ_+^L + λ_-^L
\end{eqnarray}

$Z$ ist somit exakt bekannt. Um nun den Grenzwert $L→∞$ durchzuführen, wird ausgenutzt,
dass $λ_-$ kleiner als $λ_+$ ist. Daher ist der Beitrag von $λ_-$ im Grenzfall
vernachlässigbar.

\begin{eqnarray}
Z \approx λ_+^L = \left(e^β\cosh(βB) + \sqrt{e^{2β}\cosh^2(βB) - 2\sinh(2β)}\right)^L
\end{eqnarray}