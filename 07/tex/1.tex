\section*{Aufgabe 7.1a)}
In der ersten Teilaufgabe sollte für das Potential
\begin{eqnarray}
V(x) = V_0 \Theta(ω-|x|)\left[ 1-\frac{|x|^n}{ω} \right]
\end{eqnarray}
die Streuphasen $δ_{\pm}$ berechnet und geplottet werden. Der dafür geschriebene
Code ist in \lref{streuung} dargestellt, die darin aufgerufene Funktion \texttt{ho\_fou}
ist in \lref{ho_fou} zu sehen.

\lstinputlisting[label=lst:streuung,caption={streuung.m}]{../code/streuung.m}
\lstinputlisting[label=lst:ho_fou,caption={ho\_fou.m}]{../code/ho_fou.m}

Es resultieren hierbei Plots für die drei zu verwendenden Werte von $n$, die wie
folgt dargestellt sind:
\begin{itemize}
\item für $n=1$ für \fref{n1}
\item für $n=2$ für \fref{n2}
\item für $n=10$ für \fref{n10}.
\end{itemize}

\begin{figure}[htb]
  \centering
  \includegraphics[width=0.75\columnwidth,keepaspectratio]{../tmp/71a_n1-crop}
  \caption{Streuphasen für $n=1$ für exaktes Stufenpotential und nummerische Lösung}
  \label{fig:n1}
\end{figure}

\begin{figure}[htb]
  \centering
  \includegraphics[width=0.75\columnwidth,keepaspectratio]{../tmp/71a_n2-crop}
  \caption{Streuphasen für $n=2$ für exaktes Stufenpotential und nummerische Lösung}
  \label{fig:n2}
\end{figure}

\begin{figure}[htb]
  \centering
  \includegraphics[width=0.75\columnwidth,keepaspectratio]{../tmp/71a_n10-crop}
  \caption{Streuphasen für $n=10$ für exaktes Stufenpotential und nummerische Lösung}
  \label{fig:n10}
\end{figure}

Wie man erkennen kann, stimmen für größer werdende $n$, also für immer rechteckigere
Potentiale, die exakten Lösungen und die nummerische Lösung besser überein, da
das Potential für die nummerische Lösung sich dem der exakten Lösung immer weiter angleicht.

Für $k>20$ wird die Abweichung der Lösungen unabhängig von $n$ wieder größer, wobei
davon ausgegangen wird, dass dies dadurch bewirkt wird, dass die verwendete
Näherung für große Wellenzahlen nicht mehr gültig ist.

\section*{Aufgabe 7.1b)}
In diesem Teil der Aufgabe wird die Born'sche Näherung untersucht. Die Formel, an
der sich orientiert wurde, stammt aus dem Skript.
\begin{eqnarray}
δ_+ &=& -\frac{2}{k}\int\limits_0^ω\dd{x}V(x)\cos^2(kx)\\
&=& -\frac{2 V_0}{k}\int\limits_0^ω\dd{x}(1-\frac{x}{ω})\cos^2(kx)\\
&=& \frac{V_0}{4k^3 ω} \left(-1 + \cos(2kω) - 2k^2 ω^2\right)\\
δ_- &=& -\frac{2}{k}\int\limits_0^ω\dd{x}V(x)\sin^2(kx)\\
&=& \frac{V_0}{4k^3 ω} \left(1 - \cos(2kω) - 2k^2 ω^2\right)
\end{eqnarray}
Die Integration wurde mit Hilfe des analytisch rechnenden Open-Source-Programms
\texttt{Maxima} durchgeführt. Die erhaltenen Lösungen wurden in der nächsten
Aufgabe implementiert.

\section*{Aufgabe 7.1c)}
Der entsprechende Code für die Implementierung ist im Wesentlichen in \lref{streuung}
in den Zeilen 54 bis 61 zu finden. Daraus resultiert der Plot in \fref{1c}.

\begin{figure}[htb]
  \centering
  \includegraphics[width=0.75\columnwidth,keepaspectratio]{../tmp/71c-crop}
  \caption{Vergleich der exakten Lösungen der Streuphasen mit den nummerisch
  berechneten für $n=1$}
  \label{fig:1c}
\end{figure}

Wie man erkennen kann, stimmen die Born'sche Näherung und die in a) berechnete
nummerische Lösung für etwa $5<k<20$ gut überein, während für kleinere $k$ die
Born'sche Näherung für ungerade Paritäten noch annähernd wie die nummerische Lösung
verläuft, aber für gerade Paritäten ein unphysikalisches Verhalten aufweist (starkes
Schwanken bei geringer $k$-Änderung), was durch eine Betrachtung der Reihenentwicklung
des Kosinus' erklärt werden kann.

\section*{Aufgabe 7.1d)}
In der letzten Teilaufgabe sollte die numerische Lösung für das Dreieckspotential
mit der exakten für ein Stufenpotential verglichen werden, wobei die Integrale über
beide Potentiale gleich sein sollten. Daher wurde die Höhe des Stufenpotentials auf
die Hälfte reduziert, sodass die Fläche des Stufenpotentials der des Dreieckspotentials
entspricht. Wenn die Potentiale modifiziert werden, dann ergibt sich der in \fref{1d}
dargestellte Plot.

\begin{figure}[htb]
  \centering
  \includegraphics[width=0.75\columnwidth,keepaspectratio]{../tmp/71d-crop}
  \caption{Vergleich der nummerischen Lösungen der Streuphasen für $n=1$ mit denen für eine
  exakte Potentialstufe}
  \label{fig:1d}
\end{figure}

Auch hier verlaufen die beiden betrachteten Kurven ähnlich, und für $5<k<20$ verlaufen
die Kurven sogar annähernd gleich.