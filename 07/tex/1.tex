\section*{Aufgabe 7.1a)}
In der ersten Teilaufgabe sollte für das Potential
\begin{eqnarray}
V(x) = V_0 \Theta(ω-|x|)\left[ 1-\frac{|x|^n}{ω} \right]
\end{eqnarray}
die Streuphasen $δ_{\pm}$ berechnet und geplottet werden. Der dafür geschriebene
Code ist in \lref{streuung} dargestellt, die darin aufgerufene Funktion \texttt{ho\_fou}
ist in \lref{ho_fou} zu sehen.

\lstinputlisting[label=lst:streuung,caption={streuung.m}]{../code/streuung.m}
\lstinputlisting[label=lst:ho_fou,caption={ho\_fou.m}]{../code/ho_fou.m}

Es resultieren hierbei Plots für die drei zu verwendenden Werte von $n$, die wie
folgt dargestellt sind:
\begin{itemize}
\item für $n=1$ für \fref{n1}
\item für $n=2$ für \fref{n2}
\item für $n=10$ für \fref{n10}.
\end{itemize}

\begin{figure}[htb]
  \centering
  \includegraphics[width=0.8\columnwidth,keepaspectratio]{../tmp/71a_n1}
  \caption{Streuphasen für $n=1$, }
  \label{fig:n1}
\end{figure}

\begin{figure}[htb]
  \centering
  \includegraphics[width=0.8\columnwidth,keepaspectratio]{../tmp/71a_n2}
  \caption{Streuphasen für $n=2$, }
  \label{fig:n2}
\end{figure}

\begin{figure}[htb]
  \centering
  \includegraphics[width=0.8\columnwidth,keepaspectratio]{../tmp/71a_n10}
  \caption{Streuphasen für $n=10$, }
  \label{fig:n10}
\end{figure}

Wie man erkennen kann, stimmen für größer werdende $n$, also für immer rechteckigere
Potentiale, die exakten Lösungen und die numerische Lösung besser überein.

\section*{Aufgabe 7.1b)}
In diesem Teil der Aufgabe wird die Born'sche Näherung untersucht. Die Formel, an
der sich orientiert wurde, stammt aus dem Skript.
\begin{eqnarray}
δ_+ &=& -\frac{2}{k}\int\limits_0^ω\dd{x}V(x)\cos^2(kx)\\
&=& -\frac{2 V_0}{k}\int\limits_0^ω\dd{x}(1-\frac{x}{ω})\cos^2(kx)\\
&=& \frac{V_0}{4k^3 ω} \left(-1 + \cos(2kω) - 2k^2 ω^2\right)\\
δ_- &=& -\frac{2}{k}\int\limits_0^ω\dd{x}V(x)\sin^2(kx)\\
&=& \frac{V_0}{4k^3 ω} \left(1 - \cos(2kω) - 2k^2 ω^2\right)
\end{eqnarray}
Die Integration wurde mit Hilfe des analytisch rechnenden Open-Source-Programms
\texttt{Maxima} durchgeführt. Die erhaltenen Lösungen wurden in der nächsten
Teilaufgabe implementiert.

\section*{Aufgabe 7.1c)}

\section*{Aufgabe 7.1d)}