\section*{Aufgabe 8.1)}
In der ersten Teilaufgabe war eine Bondperkolation zu implementieren. Dafür wurde
ein Feld eingeführt, dass jedem Punkt des Gitters eine Zahl zuweist, deren Bedeutung
wie folgt festgelegt wurde:
\begin{itemize}
\item $0$: passive Verbindungen zu diesem Punkt
\item $-1$: eine einzige aktive Verbindung zu diesem Punkt, zum darunter liegenden Punkt
\item $-2$: eine einzige aktive Verbindung zu diesem Punkt, zum rechts daneben liegenden Punkt
\item $-3$: zwei Verbindungen, eine zum darunter und eine zum rechts daneben liegenden Punkt
\end{itemize}

Der Code, der für diese Aufgabe geschrieben wurde, ist in \lref{bond_perk} dargestellt.

\lstinputlisting[label=lst:bond_perk,caption={bond\_perk.m}]{../code/bond_perk.m}

Die hierin aufgerufenen Funktionen \texttt{baum\_analyse.m} und \texttt{kopelman} sind in \lref{baumana} bzw. \lref{kopelman} dargestellt.

\lstinputlisting[label=lst:baumana,caption={baum\_analyse.m}]{../code/baum_analyse.m}
\lstinputlisting[label=lst:kopelman,caption={hoshen\_kopelman.m}]{../code/hoshen_kopelman.m}

Aus dem Aufruf des Hauptprogramms resultiert zum einen eine Darstellung des Gitters
inklusive aktiver Bonds (s. \fref{bonds}) und zum anderen eine Ausgabe der von den
Cluster-Algorithmen rekonstruierten Bonds, wie sie in \lref{output} dargestellt ist.

\begin{figure}[htb]
  \centering
  \includegraphics[width=0.8\columnwidth,keepaspectratio]{../tmp/bonds.png}
  \caption{Darstellung des Gitters und der aktiven Bonds}
  \label{fig:bonds}
\end{figure}

\begin{lstlisting}[caption=Ausgabe von \lref{bond_perk},label=lst:output]
octave:9> bond_perk
Bondbesetzung:
feld =

   0   0  -2  -1  -2  -1  -1  -2   0  -1
   0   0  -1   0  -1  -1  -1  -2   0  -1
   0  -2  -1  -2  -1  -2   0  -1  -1  -1
  -2  -1   0   0  -3  -3   0  -1  -3  -1
   0   0  -1  -1  -2  -1   0   0   0   0
   0  -3  -2  -3   0   0  -1  -2  -3  -1
   0  -1   0   0  -3  -3   0  -1   0  -1
  -3   0   0  -3   0  -1  -2   0  -1  -1
   0  -1   0  -1  -3   0  -1   0   0   0
   0   0   0   0   0   0  -2   0   0   0

Ergebnis, Baumsuche:
baum_feld =

    0    0    5    5    8    8    8   11   11   15
    0    0    3    5    6    8    8   12   12   15
    0    3    3    6    6    8    8   13   15   15
    1    1    3    0    6    6    6   13   15   15
    0    1    2    2    6    6    0   13   15   15
    0    2    2    2    2    6    7   14   14   14
    0    2    0    2    7    7    7    9   14   14
    2    2    0    7    7    7    9    9   16   14
    2    4    0    7    7    7   10    0   16   14
    0    4    0    7    7    0   10   10    0    0

Ergebnis, Hoshen-Kopelman:
hs_feld =

    0    0   19   19   15   15   15   18   18   12
    0    0   16   19   11   15   15   17   17   12
    0   16   16   11   11   15   15   13   12   12
   14   14   16    0   11   11    0   13   12   12
    0   14    8    8   11   11    0   13    0   12
    0    8    8    8    8   11    9    5    5    5
    0    8    0    8    2    2    9    7    0    5
    8    8    0    3    0    2    7    7    6    5
    8    4    0    3    2    2    1    0    6    5
    0    4    0    3    2    0    1    1    0    0

\end{lstlisting}

Wie man erkennen kann, ordnet die Baum-Analyse den Bonds, wie sie in der Graphik dargestellt
ist, korrekte Cluster-Nummern zu. Die Hoshen-Kopelman-Methode hingegen liefert
größtenteils, aber nicht überall korrekte Werte. Wir konnten in der uns zur Verfügung stehenden
Zeit keine Lösung der auftretenden Probleme finden.

Der größte in diesem Fall auftretende Cluster erstreckt sich über 12 Gitterpunkte.

Die Zeitmessung für die gegebenen Parameter liefert Werte von $aa$ Sekunden für die
Baumanalyse und $aa$ Sekunden für die Hoshen-Kopelman-Methode.