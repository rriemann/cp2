\section*{Aufgabe 8.1)}
In der ersten Teilaufgabe war eine Bondperkolation zu implementieren. Dafür wurde
ein Feld eingeführt, dass jedem Punkt des Gitters eine Zahl zuweist, deren Bedeutung
wie folgt festgelegt wurde:
\begin{itemize}
\item $0$: passive Verbindungen zu diesem Punkt
\item $-1$: eine einzige aktive Verbindung zu diesem Punkt, zum darunter liegenden Punkt
\item $-2$: eine einzige aktive Verbindung zu diesem Punkt, zum rechts daneben liegenden Punkt
\item $-3$: zwei Verbindungen, eine zum darunter und eine zum rechts daneben liegenden Punkt
\end{itemize}

Der Code, der für diese Aufgabe geschrieben wurde, ist in \lref{bond_perk} dargestellt.

\lstinputlisting[label=lst:bond_perk,caption={bond\_perk.m}]{../code/bond_perk.m}

Die hierin aufgerufenen Funktionen \texttt{baum\_analyse.m} und \texttt{kopelman} sind in \lref{baumana} bzw. \lref{kopelman} dargestellt.

\lstinputlisting[label=lst:baumana,caption={baum\_analyse.m}]{../code/baum_analyse.m}
\lstinputlisting[label=lst:kopelman,caption={hoshen\_kopelman.m}]{../code/hoshen_kopelman.m}

Aus dem Aufruf des Hauptprogramms resultiert zum einen eine Darstellung des Gitters
inklusive aktiver Bonds (s. \fref{bonds}) und zum anderen eine Ausgabe der von den
Cluster-Algorithmen rekonstruierten Bonds, wie sie in \lref{output} dargestellt ist.

\begin{figure}[htb]
  \centering
  \includegraphics[width=0.8\columnwidth,keepaspectratio]{../tmp/bonds.png}
  \caption{Darstellung des Gitters und der aktiven Bonds}
  \label{fig:bonds}
\end{figure}

\begin{lstlisting}[caption=Ausgabe von \lref{bond_perk},label=lst:output]
bla
\end{lstlisting}