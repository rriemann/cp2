\section*{Aufgabe 2.1d)}
Schließlich sollten die systematischen Fehler abgeschätzt werden, indem man zum
einen ε variiert und zum anderen statt \texttt{ode45} \texttt{ode23} verwendet,
was prinzipiell ungenauere Ergebnisse liefert, oder auch direkt für
\texttt{ode45} selbst andere Genauigkeiten fordert.

\subsection*{Variation von ε}
Wenn man die Breite der normalverteilten Zufallszahlen, die die Abweichung
repräsentieren, ändert, so ergeben sich auch andere Zeitpunkte, bei denen im 
Mittel die Abweichung der Trajektorien größer als 0,1 wird. Hier wurde ε so
variiert, dass es sehr klein gegenüber den Werten von $y$ und $\dot{y}$ ist und
deutlich größer als die Maschinengenauigkeit. Eine genaue Analyse des Einflusses 
der Variation ist in \lref{epsilon} zu finden.

\lstinputlisting[firstline=1,firstnumber=1,label=lst:epsilon,caption={Auswirkungen 
der Wahl von ε auf die interessanten Größen}]{../syst_errors.dat}

Wie man erkennen kann, nimmt die Zeit, die benötigt wird, um die gewünschte 
Abweichung herbeizuführen, für kleiner werdendes ε stetig zu, was auch zu
erwarten ist, da bei kleinerem ε die Ausgangspunkte und -geschwindigkeiten der
bieden Trajektorien immer weniger voneinander abweichen. Interessant zu beobachten
ist die Tatsache, dass \texttt{t\_mean} für jedes nächstkleinere ε um etwa 20
Einheiten zunimmt. Dies könnte in einer tiefergehenden Analyse begründet werden.

Weiterhin werden die 
Verteilungen immer breiter, was sich in größeren Standardabweichungen und
Vertrauensintervallen niederschlägt. Das bedeutet, dass bei die genaue Wahl der
Abweichung in Bezug auf Richtung und Betrag für kleiner werdendes ε größeren
Einfluss gewinnt.

Schon die Wahl der nächstgrößeren bzw. -kleineren Größe von ε liefert deutlich
andere Ergebnisse für \texttt{t\_mean}. Die Abweichungen betragen jeweils ca.
17 Prozent. 

Die Werte von λ schwanken von 0,1284 und 0,1360 und stimmen somit innerhalb der
Standardabweichungen, die zwischen 0.025 und 0,07 liegen, näherungsweise überein.
Die Wahl von ε hat hier also nur einen geringen Einfluss. Einzig die
Standardabweichungen selbst werden für abnehmendes ε immer kleiner.

Weiterhin ist der Einfluss der konkreten Wahl der Implementierung der numerischen
Lösung der Differentialgleichung untersucht worden, indem statt \texttt{ode45}
die Funktion \texttt{ode23} verwendet wurde. Hierfür wurden wieder 
Standardparameter verwendet. Wie man erkennen kann, stimmen die Ergebnisse bis
auf ca. 1 Prozent überein, und die Standardabweichungen nehmen ebenfalls beinah
gleiche Werte an. Die Wahl der Funktion liefert demnach nur Unterschiede im 
Prozentbereich und kann somit gegenüber dem Einfluss der Wahl von ε
vernachlässigt werden.

Schließlich wurden andere relative und absolute Genauigkeiten gefordert. Hierfür
wurden wiederum Standardparameter gewählt (\texttt{ode45}, $ε=10^{-6}$). Dies
führt den am Ende in \lref{epsilon} dargestellten Ergebnissen.

Man kann erkennen, dass man durchaus eine Größenordnung von der Standard-Toleranz
abweichen kann, ohne dass dies größere Auswirkungen auf \texttt{t\_mean} etc.
hätte. Nur wenn man die Toleranz zu groß wählt resultieren auch stärkere
Abweichungen in der Größenordnung von fünf Prozent.

Der dominante systematische Fehler ist also in in der konkreten Wahl von ε zu
suchen. Dieser beläuft sich auf ca. 17 \%, wenn man nur einen Bereich von $ε = 
10^{-5}$ bis $ε = 10^{-7}$ zulässt.

Die durch die verschiedenen Untersuchungen erhaltenen λ-Werte variieren zwischen
0,127 ± 0,042 und 0,13599 ± 0,068 und sind somit im gesamten Bereich der
Veränderungen der Parameter näherungsweise konstant geblieben.

Eine weitere Erhöhung der Anzahl der Wiederholungen würde zu einer Erhöhung der 
statistischen Genauigkeit führen. Da die statistischen Fehler, repräsentiert
durch das Vertrauensintervall, im Bereich von ca. zwei Prozent liegt, dominiert
insgesamt der systematische Fehler. Demnach ist es nicht sinnvoll, noch mehr 
Versuche durchzuführen, um den gesamten Fehler deutlich zu senken.