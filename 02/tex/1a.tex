\section*{Aufgabe 2.1a)}
In dieser Aufgabe war der gegebene Quelltext zu verwenden und so abzuändern, 
dass zwei Trajektorien mit beinahe gleichen Startbedingungen daraufhin
untersucht werden können, ab welchem Zeitpunkt die Differenz der Orte größer als
0,1 wird (siehe Aufgabenstellung). Diese Messung wird 200mal wiederholt, um
statistische Aussagen über diesen Zeitpunkt treffen zu können.

Der Octave-Quelltext dafür wird hier gezeigt:
\lstinputlisting[firstline=1,firstnumber=1,label=lst:chaos,caption={chaos.m}]{../chaos.m}

Wenn man diesen ausführt, so resultiert unter anderem der folgende Output:
\begin{lstlisting}[caption=Output unseres Programms,label=lst:output]
t_mean =  93.263
t_std =  26.290
t_conf_int =  1.8590
lamda_mean =  0.13447
lamda_std =  0.042518
\end{lstlisting}