\section*{Aufgabe 1}
In dieser Übung sollte untersucht werden, welche Werte die kritischen Exponenten
haben, wofür Punktgitter verschiedener Größe zufällig punktweise aktiviert wurden
und anschließend in mehrere Hinsicht analysiert wurden. Zunächst wurden die Cluster
mit Hilfe des Hoshen-Kopelman-Algorithmus rekonstruiert, die Existenz
von perkolierenden Clustern überprüft und anschließend die Wahrscheinlichkeit, dass
ein aktiver Punkt zu diesem speziellen Cluster gehört, berechnet. Des Weiteren
wurde die mittlere Clustergröße berechnet. Bei der Berechnung dieser Größen wurde
sich an die Formeln aus dem Skript gehalten.

Der C-Code, der dies bewerkstelligt, ist in \lref{code}, die aufgerufene
Funktion zur Anwendung des Hoshen-Kopelman-Verfahrens in \lref{hk}, der Zufallszahlen
generierende Code in \lref{rand} dargestellt, wobei die zugehörigen Headerdateien 
direkt nach der jeweiligen Datei selbst dargestellt sind.

\lstinputlisting[label=lst:code,caption={punkt\_perk.c}]{../code/punkt_perk.c}
\lstinputlisting[label=lst:hk,caption={cluster\_analyse.c}]{../code/cluster_analyse.c}
\lstinputlisting[caption={cluster\_analyse.h}]{../code/cluster_analyse.h}
% \lstinputlisting[label=lst:rand,caption={R250.c}]{../code/R250.c}
\lstinputlisting[caption={R250.h}]{../code/R250.h}

Schließlich wurde der Code mit Hilfe des in \lref{makefile} dargestellten Makefiles
kompiliert und die Ergebnisse in eine Datei geschrieben, um später mit Gnuplot
visualisiert werden zu können. Dazu wurde das Skript aus \lref{plot} verwendet.

\lstinputlisting[language=make,label=lst:makefile,caption={Makefile}]{../code/Makefile}
\lstinputlisting[language=Gnuplot,label=lst:plot,caption={plot.gp}]{../code/plot.gp}

Wenn man die Ergebnisse unseres Codes wie gefordert doppelt-logarithmisch darstellt
und außerdem die jeweiligen Exponenten des vermuteten zugrundeliegenden Potenzgesetzes
mit Hilfe eines Fits bestimmt, so ergeben sich die beiden Plots in \fref{P} und
\fref{S} sowie folgende Werte für die Exponenten:

\begin{figure}[htb]
  \centering
  \includegraphics[width=0.8\columnwidth,keepaspectratio]{../tmp/p_inf}
  \caption{Wahrscheinlichkeit $P_{\infty}$ der Zugehörigkeit eines aktiven Punktes zum perkolierenden Cluster in Abhängigkeit von der Größe des Gitters}
  \label{fig:P}
\end{figure}

\begin{figure}[htb]
  \centering
  \includegraphics[width=0.8\columnwidth,keepaspectratio]{../tmp/s}
  \caption{Mittlere Clustergröße $S$ in Abhängigkeit von der Größe des Gitters}
  \label{fig:S}
\end{figure}

\begin{table}[htbp]
\centering
\setlength{\tabcolsep}{14pt}
\begin{tabular*}{0.5\columnwidth}{lll}
\toprule
& {$β/ν$} & {$γ/ν$}\\
\midrule
\textit{Fit} & $0.180\pm0.024$ & $1.772\pm0.024$ \\
\textit{Sollwert} & $0.104$ & $1.792$ \\
\bottomrule
\end{tabular*}
\label{tab:res}
\caption{Vergleich unserer Fitergebnisse mit den Sollwerten für die Exponenten}
\end{table}

Wie man erkennen kann, stimmen speziell die Ergebnisse des Fits für die mittlere Clustergröße $S$
sehr gut mit den gegebenen Werten überein. Auch für den Fit an $P_{\infty}$ in Abhängigkeit von
$L$ resultieren Ergebnisse, die in der gleichen Größenordnung liegen wie der gewünschte Wert, aber hier
tritt eine Abweichung von ca. 58\% auf, wobei uns nicht klar ist, woran dies liegen könnte. Wir haben
festgestellt, dass der Zufallszahlengenerator starke Abhängigkeiten von der konkreten Wahl des Seeds
zeigt. In einem späteren Versuch könnte man diesen durch einen in der Programmiersprache selbst implementierten
bzw. standardisierten Generator verwenden.
