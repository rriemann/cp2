\section*{Aufgabe 1}
In dieser Übung sollte untersucht werden, welche Werte die kritischen Exponenten
haben, wofür Punktgitter verschiedener Größe zufällig punktweise aktiviert wurden
und anschließend in mehrere Hinsicht analysiert wurden. Zunächst wurden die Cluster
mit Hilfe des Hoshen-Kopelman-Algorithmus rekonstruiert, die Existenz
von perkolierenden Clustern überprüft und anschließend die Wahrscheinlichkeit, dass
ein aktiver Punkt zu diesem speziellen Cluster gehört, berechnet. Des Weiteren
wurde die mittlere Clustergröße berechnet.

Der C-Code, der dies bewerkstelligt, ist in \lref{code}, die aufgerufene
Funktion zur Anwendung des Hoshen-Kopelman-Verfahrens in \lref{hk}, der Zufallszahlen
generierende Code in \lref{rand} dargestellt, wobei die zugehörigen Headerdateien 
direkt nach der jeweiligen Datei selbst dargestellt sind.

\lstinputlisting[label=lst:code,caption={punkt\_perk.c}]{../code/punkt_perk.c}
\lstinputlisting[label=lst:hk,caption={cluster\_analyse.c}]{../code/cluster_analyse.c}
\lstinputlisting[caption={cluster\_analyse.h}]{../code/cluster_analyse.h}
\lstinputlisting[label=lst:rand,caption={R250.c}]{../code/R250.c}
\lstinputlisting[caption={R250.h}]{../code/R250.h}

Schließlich wurde der Code mit Hilfe des in \lref{makefile} dargestellten Makefiles
kompiliert und die Ergebnisse in eine Datei geschrieben, um später mit Gnuplot
visualisiert werden zu können.

\lstinputlisting[label=lst:makefile,caption={Makefile}]{../code/Makefile}