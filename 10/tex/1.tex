\section*{Aufgabe 1}
In dieser Aufgabe sollte das Volumen und das Trägheitsmoment einer (abgeflachten)
Kugel mit Hilfe einer Monte-Carlo-Simulation näherungsweise bestimmt werden. Im Fall
der abgeflachten Kugel wurde die $x$-Koordinate so eingeschränkt, dass sie nur Werte 
zwischen $-a$ und $+a$ annehmen kann, mit $a\le R$.

Zunächst wurde das exakte Kugelvolumen für beide $a$-Werte und das Trägheitsmoment
für $a=R$ berechnet, um eine Schätzung für die Standardabweichungen zu erhalten.
Diese wurden nach der Formel
\begin{eqnarray}
σ = \sqrt{\langle x²\rangle - \overline{x}²}
\label{eqn:sigma}
\end{eqnarray}
oder nach Gl. (7.6) und (7.7) im Skript berechnet (beide äquivalent). Für den Fall $a\ne R$ wurde statt der exakten Werte $\overline{x}$ der numerisch
gebildete Mittelwert $\langle x \rangle$ verwendet.

Das Kugelvolumen wurde wie folgt bestimmt:
\begin{eqnarray}
p &=& \frac{V_{Kugel}}{V_{G}} = \frac{N_{Kugel}}{N_G}\\
V_{Kugel} &=& pV_G = \frac{N_{Kugel}}{N_G}\cdot (2R)^2\cdot 2a\\
&=& 8R^2a\frac{N_{Kugel}}{N_G}
\end{eqnarray}
Hierbei sind $V_G$ das Volumen des Gebiets $G$ und $N_G$ die Anzahl der simulierten
Punkte in diesem Gebiet.

Das Trägheitsmoment wurde nach der Formel
\begin{eqnarray}
J_z &=& \intρ(\vec{r})r^2\dd{\vec{r}}
\end{eqnarray}
numerisch berechnet. Der Wert für die Kugel wurde der Literatur entnommen.

Der Code, der die Aufgabenstellung bearbeitet, ist in \lref{kugel} dargestellt.

\lstinputlisting[label=lst:kugel,caption={kugel.m}]{../code/kugel.m}

Der resultierende Output ist hier in Form einer Tabelle in \tref{erg} dargestellt.

\begin{table}[htbp]
\centering
\setlength{\tabcolsep}{14pt}
\begin{tabular*}{0.5\columnwidth}{lll}
\toprule
$a = 2$ & {$V$} & {$J_p$}\\
\midrule
\textit{MC} & $ 33.3\pm0.2 $ & $ 53.8\pm0.3 $ \\
\textit{exakt} & $ 33.5 $ & $ 53.6 $ \\
\midrule
$a = 1$ & {$V$} & {$J_p$}\\
\midrule
\textit{MC} & $ 23.1\pm0.2 $ & $ 28.7\pm0.3 $ \\
\textit{exakt} & $ 23.0 $ & - \\
\bottomrule
\end{tabular*}
\caption{Vergleich von numerisch ermittelten (MC) und exakten Werten}
\label{tab:erg}
\end{table}

Wie man erkennen kann, stimmen die numerisch ermittelten Werte mit den exakt bekannten
im Rahmen der gemessenen Fehler gut überein.

\section*{Aufgabe 2}
In dieser Aufgabe war die Akzeptanz eines Detektors zu berechnen. Der entsprechende
Code ist in \lref{detektor} dargestellt.

\lstinputlisting[label=lst:detektor,caption={detektor.m}]{../code/detektor.m}

Das resultierende Ergebnis für die Akzeptanz $A$ lautet dabei:
\begin{eqnarray}
A = (83.04 \pm 0.08)\%
\end{eqnarray}

Der Fehler wurde gemäß \eref{sigma} berechnet, wobei auch hier mit numerisch
berechneten Mittelwerten statt exakten Werten gerechnet wurde.

Eine kurze Erläuterung der verwendeten Größen sowie zugehörige Skizzen sind als
Anhang beigefügt.