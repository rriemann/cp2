\section*{Aufgabe 4.1a)}
Das Thema dieser Übung ist die Fouriertransformation. Diese kann mit numerischen
Methoden auf einfache Weise durchgeführt werden. Die Grundlagen wurden dem Skript
entnommen (Gl. (3.5) und (3.6)):
\begin{eqnarray}
f(x) &=& \frac{1}{L}\sum_k e^{ikx}\tilde{f}(k)\\
\tilde{f}(k) &=& a\sum_y e^{-iky} f(y)
\end{eqnarray}
Diese beiden Formeln beschreiben die Hin- und Rücktransformation. Im ersten Teil
dieser Aufgabe waren diese Transformationen in zwei Funktionen zu implementieren,
um gegebene, diskrete Funktionswerte in diskrete Werte der transformierten 
Funktion umzuwandeln. Die Quellcodes dafür sind in \lref{trafo} und
\lref{ruecktrafo} dargestellt.

\lstinputlisting[firstline=1,firstnumber=1,label=lst:trafo,caption={trafo.m}]{../code/trafo.m}

\lstinputlisting[firstline=1,firstnumber=1,label=lst:ruecktrafo,caption={ruecktrafo.m}]{../code/ruecktrafo.m}

Um zu testen, ob die Transformationen wie erwartet funktionieren, wurde mit einfachen
Vektoren getestet, ob die Vektoren nach einer Hin- und einer Rücktransformation
invariant bleiben. Der aufrufende Code sieht wie folgt aus:

\lstinputlisting[firstline=1,lastline=13,firstnumber=1,label=lst:fouriertest,caption={fourier.m}]{../code/fourier.m}

Der resultierende Output ist hier dargestellt: 

\begin{lstlisting}[caption=Output des Beispielaufrufs,label=lst:fouriertestoutput]
Vektor eins: 
v1 =

   1
   2
   3
   4
   5

f1 =

   3.00000 + 0.00000i
  -0.50000 - 0.68819i
  -0.50000 - 0.16246i
  -0.50000 + 0.16246i
  -0.50000 + 0.68819i

ergebnis =

   1.0000 + 0.0000i
   2.0000 + 0.0000i
   3.0000 - 0.0000i
   4.0000 - 0.0000i
   5.0000 - 0.0000i

Vektor zwei: 
v2 =

  -0.10000
   0.75000
   3.14159

f2 =

   1.26386 + 0.00000i
  -0.68193 - 0.69039i
  -0.68193 + 0.69039i

ergebnis =

  -0.10000 + 0.00000i
   0.75000 + 0.00000i
   3.14159 + 0.00000i
\end{lstlisting}

Wie man erkennen kann, stimmen die Ausgangsvektoren mit denen nach der Hin- und
Rücktransformation überein. Man kann also annehmen, dass die Transformationen wie
gewünscht funktionieren.

\section*{Aufgabe 4.1b)}
Im zweiten Teil der Aufgabe war eine Sinusfunktion aus dem $t$- in den ω-Raum zu
transformieren, also $f(t) = \sin(ω_0t) → \tilde{f}(ω)$. Der entsprechende Code
ist in \lref{sinfkt} dargestellt.

\lstinputlisting[firstline=17,lastline=31,firstnumber=17,label=lst:sinfkt,caption={fourier.m}]{../code/fourier.m}

Die entstehende Grafik ist in \fref{sintrafo} dargestellt.

\begin{figure}[htb]
\centering
  \includegraphics[width=0.9\textwidth,keepaspectratio]{../code/sintrafo-crop}
  \caption{In den ω-Raum transformierte von $\sin(ω_0t)$. Auf der Abszisse sind
  die ω-Werte aufgetragen, auf der Ordinate die Funktionswerte.}
  \label{fig:sintrafo}
\end{figure}


\section*{Aufgabe 4.1c)}