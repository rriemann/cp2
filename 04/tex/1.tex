\section*{Aufgabe 4.1a)}
Das Thema dieser Übung ist die Fouriertransformation. Diese kann mit numerischen
Methoden auf einfache Weise durchgeführt werden. Die Grundlagen wurden dem Skript
entnommen (Gl. (3.5) und (3.6)):
\begin{eqnarray}
f(x) &=& \frac{1}{L}\sum_k e^{ikx}\tilde{f}(k)\\
\tilde{f}(k) &=& a\sum_y e^{-iky} f(y)
\end{eqnarray}
Diese beiden Formeln beschreiben die Hin- und Rücktransformation. Im ersten Teil
dieser Aufgabe waren diese Transformationen in zwei Funktionen zu implementieren,
um gegebene, diskrete Funktionswerte in diskrete Werte der transformierten 
Funktion umzuwandeln. Die Quellcodes dafür sind in \lref{trafo} und
\lref{ruecktrafo} dargestellt.

\lstinputlisting[firstline=1,firstnumber=1,label=lst:trafo,caption={trafo.m}]{../code/trafo.m}

\lstinputlisting[firstline=1,firstnumber=1,label=lst:ruecktrafo,caption={ruecktrafo.m}]{../code/ruecktrafo.m}

Um zu testen, ob die Transformationen wie erwartet funktionieren, wurde mit einfachen
Vektoren getestet, ob die Vektoren nach einer Hin- und einer Rücktransformation
invariant bleiben. Der aufrufende Code sieht wie folgt aus:

\lstinputlisting[firstline=1,lastline=13,firstnumber=1,label=lst:fouriertest,caption={fourier.m}]{../code/fourier.m}

Der resultierende Output ist hier dargestellt: 

\begin{lstlisting}[caption=Output des Beispielaufrufs,label=lst:fouriertestoutput]
Vektor eins: 
v1 =

   1
   2
   3
   4
   5

f1 =

   3.00000 + 0.00000i
  -0.50000 - 0.68819i
  -0.50000 - 0.16246i
  -0.50000 + 0.16246i
  -0.50000 + 0.68819i

ergebnis =

   1.0000 + 0.0000i
   2.0000 + 0.0000i
   3.0000 - 0.0000i
   4.0000 - 0.0000i
   5.0000 - 0.0000i

Vektor zwei: 
v2 =

  -0.10000
   0.75000
   3.14159

f2 =

   1.26386 + 0.00000i
  -0.68193 - 0.69039i
  -0.68193 + 0.69039i

ergebnis =

  -0.10000 + 0.00000i
   0.75000 + 0.00000i
   3.14159 + 0.00000i
\end{lstlisting}

Wie man erkennen kann, stimmen die Ausgangsvektoren mit denen nach der Hin- und
Rücktransformation überein. Man kann also annehmen, dass die Transformationen wie
gewünscht funktionieren.

\section*{Aufgabe 4.1b)}
Im zweiten Teil der Aufgabe war eine Sinusfunktion aus dem $t$- in den ω-Raum zu
transformieren, also $f(t) = \sin(ω_0t) → \tilde{f}(ω)$. Der entsprechende Code
ist in \lref{sinfkt} dargestellt.

\lstinputlisting[firstline=17,lastline=31,firstnumber=17,label=lst:sinfkt,caption={fourier.m}]{../code/fourier.m}

Die entstehende Grafik ist in \fref{sintrafo} dargestellt.

\begin{figure}[htb]
\centering
  \includegraphics[width=0.7\textwidth,keepaspectratio]{../code/sintrafo-crop}
  \caption{In den ω-Raum transformierte von $\sin(ω_0t)$. Auf der Abszisse sind
  die ω-Werte aufgetragen, auf der Ordinate die Funktionswerte.}
  \label{fig:sintrafo}
\end{figure}

Wir hätten erwartet, dass es zwei diskrete Peaks gibt, da dies einer Zerlegung 
einer Sinusfunktion in $e$-Funktionen entspricht ($\sin(ωt) = 1/(2i)\cdot(e^{iωt}
- e^{-iωt}) → \pm ω$ sind die entsprechenden Werte). Hier sind allerdings viele 
Funktionswerte ungleich null, was auf Ungenauigkeiten der Rechnung beruhen wird. 
Diese Werte sind aber ca. vier bis fünf Größenordnungen kleiner als die der Maxima.
So könnte beispielsweise die Maschinengenauigkeit von Bedeutung sein, und auch die
numerische Implementation der Sinus- und Exponentialfunktionen ist nicht exakt. 
Dies könnte durchaus für die Abweichungen verantwortlich sein.

Für die Symmetrisierung wurde die Periodizität des Sinus ausgenutzt ($\sin(-2π)
= \sin(0) = …$), um eine Verschiebung der Funktion zu anderen ω-Werte durchzuführen.

\section*{Aufgabe 4.1c)}
Der letzte Teil der ersten Aufgabe befasst sich mit der Sinusfunktion, bei der 
entweder eine Dämpfung hinzugefügt oder der Wert von $ω_0$ auf $4.5\cdot 2π/T$
geändert wurde. Der Code dafür ist in \lref{1c} dargestellt.

\lstinputlisting[firstline=36,lastline=47,firstnumber=36,label=lst:1c,caption={fourier.m}]{../code/fourier.m}

Für die Dämpfung ergibt sich die Darstellung in \fref{daempfung}, für den
veränderten $ω_0$-Wert die in \fref{w0}

\begin{figure}[htb]
\centering
  \includegraphics[width=0.7\textwidth,keepaspectratio]{../code/daempfung-crop}
  \caption{Transformierte der gedämpften Sinusfunktion}
  \label{fig:daempfung}
\end{figure}

\begin{figure}[htb]
\centering
  \includegraphics[width=0.7\textwidth,keepaspectratio]{../code/sintrafo2-crop}
  \caption{Transformierte der Sinusfunktion mit verändertem $ω_0$}
  \label{fig:w0}
\end{figure}

Für die Dämpfung lässt sich erkennen, dass nun ein deutlich größeres Spektrum
von ω-Werten relevante Funktionswerte beiträgt. Dennoch ist die grobe Struktur der Grafik 
der aus \fref{sintrafo} ähnlich. Man kann also feststellen, dass der Einfluss der
Sinusfunktion noch stark vorherrscht.

In \fref{w0} erkennt man ebenfalls noch die Ursprungsfunktion aus \fref{sintrafo},
doch auch hier tragen mehr ω-Werte zur Verteilung bei. 

Die Tatsache, dass in beiden Fällen die Peaks „verbreitert“ werden, also mehr 
ω-Werte beitragen, lässt sich dadurch erklären, dass es sich um diskrete
Funktionen handelt, sodass eine Änderung von $ω_0$ bei konstant gehaltener 
Schrittweite und Gesamtlänge $T$ eine Änderung der Zerlegung in $e$-Funktionen
bewirkt.