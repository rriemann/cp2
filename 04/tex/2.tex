\section*{Aufgabe 4.2)}

\lstinputlisting[firstline=51,label=lst:aufg42,caption={fourier.m}]{../code/fourier.m}

\begin{figure}[htb]
\centering
  \includegraphics[width=0.7\textwidth,keepaspectratio]{../code/chaosdata_trafo-crop}
  \caption{In den ω-Raum transformierte Funktion aus den Daten von \texttt{chaosdata.mat}. Auf der Abszisse sind
  die ω-Werte aufgetragen, auf der Ordinate die Funktionswerte.}
  \label{fig:chaos}
\end{figure}

\begin{figure}[htb]
\centering
  \includegraphics[width=0.7\textwidth,keepaspectratio]{../code/chaosdata_trafo_log-crop}
  \caption{Equivalente, logarthmische Darstellung von \fref{chaos}}
  \label{fig:chaoslog}
\end{figure}

Die chaotische Trajektorie aus Kap. 3 wurde mittels der zuvor eingeführten
Transformation vom Ortsraum in den Impulsraum übersetzt (siehe \lref{aufg42}). Aus \fref{chaos} lässt
sich entnehmen, dass nur ein sehr enges Frequenzspektrum zur Beschreibung der
Trajektorie notwendig ist. Aus dem logarthmischen Plot in \fref{chaoslog} kann
entnommen werden, dass die Werte außerhalb der Intervalls um den Ursprung zwar nicht
null sind, doch mit zunehmenden Betrag von ω schnell abfallen.

Man kann also über die chaotische Bewegung nun gewisse Aussagen anhand der
Fourier-Analyse treffen. So schwingt die Trajektorie zumeist mit einer
Winkelfrequenz von $|ω| < 10$. Größere Werte von ω können im wesentlichen
unberücksichtigt bleiben. Die Wahrscheinlichkeit für eine sehr hohe Schwingungsabfolge
($|ω| > 10$) ist also sehr gering und kommt im betrachteten Datensatz
praktisch nicht vor. Aus den vorhandenen Maxima lassen sich Frequenzen mit einer
hohen Relevanz erkennen.